As electronics devices scale to sub 10 nanometer lengths, the distinction
between ``device'' and ``leads'' or ``electrodes'' becomes blurred. Here, the
interaction between device and leads is studied in a simple model of a
molecular tunnel junction. Using a complex absorbing potential, we are able to
reproduce the single-particle energy levels of the device region including a
description of the effects of the ``semi-infinite'' leads. With this approach,
we are able to model the effect of coupling of a quantum device to leads and to
systematically study the effect of many-electron interactions between the
device and lead regions. Varying the device-lead coupling, the effect of
electron correlation on energy shifts and lifetimes of electronic states on the
device region is studied by systematically increasing the electron correlation
or ``many-electron interactions''. To achieve the treatment of the electronic
interactions, we apply the method of configuration interaction. We find that
the prediction of the electronic states of a device region is sensitive to both
the amount of device-lead coupling and to electron correlation. The two effects
mix in a complicated way implying that detailed treatments of the electronic
structure of nanoscale devices are required to predict electronic behaviour
such as charge transport and photoexcitations in a molecular junction.
