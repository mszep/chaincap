As electronics devices scale to sub--10 nanometer lengths, the distinction
between `device' and `electrodes' becomes blurred.
Here, we study a simple model of a molecular tunnel junction, consisting
of an atomic gold chain partitioned into left and right electrodes, and a
central `molecule'.
Using a complex absorbing potential, we are able to reproduce the
single-particle energy levels of the device region including a description
of the effects of the ``semi-infinite'' electrodes.
We then use the method of configuration interaction to explore the effect
of correlations on the system's quasiparticle peaks.
We find that when excitations on the leads are not included, the device's
\ac{HOMO} and \ac{LUMO} quasiparticle peaks when including correlation
are bracketed by their respective values in the Hartree-Fock (Koopmans)
and \dscf approximations.
In contrast, when excitations on the leads are included, the bracketing
property no longer holds and both the positions and the lifetimes of the
quasiparticle levels change considerably, indicating that the combined
effect of coupling and correlation is to alter the quasiparticle spectrum
significantly from that of an isolated molecule.
