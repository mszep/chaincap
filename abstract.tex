As electronics devices scale to sub 10 nanometer lengths, the distinction
between `device' and `leads' or `electrodes' becomes blurred.
Here, we study a simple model of a molecular tunnel junction, consisting
of an atomic gold chain partitioned into left and right electrodes, and a
central `molecule'.
Using a complex absorbing potential, we are able to reproduce the
single-particle energy levels of the device region including a description
of the effects of the ``semi-infinite'' leads.
We then use the method of configuration interaction to explore the effect
of correlations on the system's quasiparticle peaks.
We find that when excitations on the leads are not included, the devices'
\ac{HOMO} and \ac{LUMO} quasiparticle peaks when including correlation
are bracketed by their respective values at the Hartree-Fock (Koopmans)
and \dscf level of theory.
Increasing the coupling strength (by decreasing the molecule-lead
separation) produces wider peaks, corresponding to shorter lifetimes.
In contrast, including excitations on the leads has significant effects
on the transmission (and hence the conductivity) of the device. The
bracketing property no longer holds and both the positions and the lifetimes
of the quasiparticle levels shift significantly.
