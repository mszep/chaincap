\section{Many-Body Analysis}

We start our analysis by considering the more weakly coupled system (45
nm separation). The transmission curves obtained for the \ac{HOMO} at
different levels of theory are plotted in figure (whatever). Firstly,
we can represent the wave function as a single determinant. Since the
only difference between this and the single particle case described above
is the basis, it is not surprising to see good agreement, especially in
the position of the peak. Next, we can approximate the wave function by
taking the same reference determinant as above, and also including all
of its single excitations. We can expect that the results from this are
analogous to a delta-SCF procedure, since by Thouless' theorem this can
be described as a single determinant in a different basis. Finally, we
can include more correlations by running the mcci procedure, first at
cmin=3e-3, and then at cmin=1e-3, which generally corresponds to moderate
correlation. As expected from chemical intuition, CI singles gives the
smallest gap and single determinants the largest, with the correlated
peaks in the middle and the gap closing with increasing correlation.
The width of the peaks in the different many-body approximations stays
nearly constant, implying that broadening (in this case corresponding
to a state lifetime of approximately XXX) is accurately captured at the
SCF (single determinant) level for this system.

We now repeat the analysis for the more strongly coupled system, results
of which are plotted in figure (whatever2). The same picture emerges,
with the CI singles case having the smallest gap, and the single determinant
the largest, although at the \ac{HOMO} now the weakly correlated (cmin=3e-3)
case overlaps with the single determinant case. However, when decreasing
the cmin parameter (increasing correlation), the peak shifts back up,
reducing the gap. With respect to the broadenings, we see a similar
picture as for the previous system; increasing correlation has no
significant effect on the level broadening.

Finally, we investigate the effect of including excitations on the leads.
Until this point, we have always constrained our CI space to \acp{CSF}
which represent excitations from and to orbitals localized on the molecule.
Lead excitations have been previously shown to have significant effects
on current flow through two-level model systems
\cite{galperin_nitzan2006leadexcitations}. 
The first notable feature is found in the CI singles case, where the
\ac{LUMO} peak is much broader than all other peaks studied. This can
be directly related to the fact that the $N+1$-electron contains two
(instead of one) significant CSFs, one of which has the extra electron
in a lead orbital. However, upon including more CSFs in the mcci procedure,


then device region
